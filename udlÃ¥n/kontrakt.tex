\documentclass[a4paper]{article}
\usepackage[utf8]{inputenc}
\usepackage[T1]{fontenc}
\usepackage[english]{babel}
\usepackage{microtype}
\usepackage{palatino}
\usepackage{mathtools}
\usepackage{amssymb}
\usepackage{pdfpages}
\usepackage{fancyhdr}
\usepackage{underscore}
\usepackage[lmargin=2cm,rmargin=2cm,tmargin=1cm,bmargin=1cm]{geometry}
\usepackage{multicol}
\usepackage[hidelinks]{hyperref}

\setlength{\parindent}{0mm}
\setlength{\parskip}{3mm}
\pagestyle{empty}

\begin{document}

\begin{center}
\Huge Udlånsaftale\\
\small Udgave: 15. august 2016
\end{center}

Nærværende atale indgås mellem låner og Kantineforening ved Datalogisk Institut
vedrørende udlån af kantinen i Universitetsparken 1, 2100 København Ø.
Foreningen stiller kantinens faciliteter og inventar til rådighed for låner i
den i afsnit 1 angivne periode, under de i afsnit 3 angivnge betingelser.


\section{Låneperiode}

Format: 20YY-MM-DD.  Udlånet løber fra  20 \underline{\hspace{1cm}} -
\underline{\hspace{1cm}} - \underline{\hspace{1cm}} til 20 \underline{\hspace{1cm}} -
\underline{\hspace{1cm}} - \underline{\hspace{1cm}}.


\section{Depositum}

Låner har betalt depositum af \underline{\hspace{2cm}} kr.  Depositum
tilbagebetales efter udlånets ophør forudsat betingelserne i afsnit 3 er
overholdt.


\section{Lånebetingelser}

\begin{enumerate}
\item Låner hæfter for skader på kantinens og instituttets inventar og
bygninger, der måtte være foretaget af låner eller låners gæster.
\item kantinen er omfattet af statens rygepolitik, dvs. der må \emph{ikke} ryges
indendørs på instituttet.
\item Det er låners ansvar at uvedkommende ikke opholder sig i kantinen under
udlånet.
\item Det er låners ansvar at låners gæster ikk opholder sig på arealer de
ellers ikke vile have adgang til på instituttet.  Låner og låners gæster må
\emph{ikke} lukke uvedkommende ind på sådanne arealer.
\item Låner har modtaget en kopi af retningslinjerne for oprydning og rengøring
efter udlån og accepterer disse.
\item Der må \emph{ikke} soves på instituttet.
\item Foreningen kan ikke garantere at de varer der sædvanligvis sælges i
kantinen vil være tilgængelige under udlånet.
\item Hvis arrangementet begrænser adgang til kantinen for instituttets
studerende og ansatte, skal dette annonceres via fysiske sedler hængt på
indgangene til kantinen, og gerne via email på \url{sci-diku-social@list.ku.dk}
(rammer dog ikke alle) og sociale medier (rammer heller ikke alle).
\item Medlemmer af foreningens bestyrelse har til enhver tid adgang til
kantinen og kan afbryde udlånet uden varsel.
\end{enumerate}

\section{Underskrifter}


\begin{multicols}{2}
\textbf{Som låner:}

\newcommand{\skrivher}[1]{\underline{\hspace{7.5cm}}\\[-1mm]{\scriptsize #1}}

\skrivher{DATO}

\skrivher{UNDERSKRIFT}

\skrivher{NAVN}

\skrivher{ADRESSE}

\skrivher{EMAIL}

\skrivher{TELEFONNUMMER}

\columnbreak

\textbf{På vegne af foreningens bestyrelse:}

\skrivher{DATO}

\skrivher{UNDERSKRIFT}

\skrivher{NAVN}

\end{multicols}


\end{document}
